\usepackage{indentfirst} 
\setlength\parindent{1.25cm}
 \usepackage{stix}
 \usepackage{array}
 
 \newcommand{\nocont}[1]{\begin{center}
 		\large\bfseries{#1}
 \end{center} }

\usepackage[utf8]{inputenc}    % Кодировка UTF-8
\usepackage{cmap} % Улучшенный поиск русских слов в полученном pdf-файле
\usepackage[T2A]{fontenc} % Поддержка русских букв
\usepackage[english,russian]{babel} % Русский и английский язык
\usepackage[14pt]{extsizes}
\usepackage{amsmath,amsfonts,amssymb}
\usepackage{graphicx}          
\usepackage{geometry}          
\geometry{
	left=30mm,
	right=10mm,
	top=20mm,
	bottom=20mm
}
\usepackage{pgfplots}
\usepackage{tocloft}
\usepackage{tempora}
\usepackage{setspace}
\usepackage{pdfpages}
\usepackage{enumerate,letltxmacro}
\usepackage{threeparttable}
\usepackage{hyperref}
\usepackage{flafter}
\usepackage{enumitem}
\usepackage{multirow}

\usepackage[figure,table]{totalcount}
\usepackage{float}
\usepackage{tabularx}

% Гиперссылки
\hypersetup{pdfborder=0 0 0}

% Для оформления заголовков разделов
\usepackage{titlesec} 
\makeatother
\RequirePackage{titlesec}
\titleformat{\chapter}[block]{\hspace{\parindent}\large\bfseries}{\thechapter}{0.5em}{\large\bfseries\raggedright}
\titleformat{name=\chapter,numberless}[block]{\hspace{\parindent}}{}{0pt}{\large\bfseries\centering}
\titleformat{\section}[block]{\hspace{\parindent}\large\bfseries}{\thesection}{0.5em}{\large\bfseries\raggedright}
\titleformat{\subsection}[block]{\hspace{\parindent}\large\bfseries}{\thesubsection}{0.5em}{\large\bfseries\raggedright}
\titleformat{\subsubsection}[block]{\hspace{\parindent}\large\bfseries}{\thesubsection}{0.5em}{\large\bfseries\raggedright}
\titlespacing{\chapter}{12.5mm}{-22pt}{10pt}
\titlespacing{\section}{12.5mm}{10pt}{10pt}
\titlespacing{\subsection}{12.5mm}{10pt}{10pt}
\titlespacing{\subsubsection}{12.5mm}{10pt}{10pt}

\usepackage{xcolor}
\usepackage{setspace} 
\onehalfspacing 

\usepackage{caption}

\addto\captionsrussian{\renewcommand{\figurename}{Рисунок}}
\DeclareCaptionLabelSeparator{dash}{~---~}
\captionsetup{labelsep=dash}

\captionsetup[figure]{justification=centering,labelsep=dash}
\captionsetup[table]{labelsep=dash,justification=raggedright,singlelinecheck=off}


\usepackage{listings}
% Настройка листинга
\lstset{%
	language=SQL,   		% выбор языка для подсветки	
	basicstyle=\ttfamily\footnotesize,		% размер и начертание шрифта для подсветки кода
	frame=single,			% рисовать рамку вокруг кода
	tabsize=4,			    % размер табуляции по умолчанию равен 4 пробелам
	captionpos=t,			% позиция заголовка вверху [t] или внизу [b]
	breaklines=true,		   % Перенос длинных строк		
	breakatwhitespace=true,	   % переносить строки только если есть пробел
	escapeinside={\#*}{*)},		% если нужно добавить комментарии в коде
	backgroundcolor=\color{white},
}



%\usepackage{showframe} %отображать границы полей страницы
\usepackage{ragged2e}
\usepackage{float}


\newcommand{\ssr}[1]{\begin{center}
		\large\bfseries{#1}
	\end{center} \addcontentsline{toc}{chapter}{#1}  }


\makeatletter
\renewcommand\LARGE{\@setfontsize\LARGE{22pt}{20}}
\renewcommand\Large{\@setfontsize\Large{20pt}{20}}
\renewcommand\large{\@setfontsize\large{16pt}{20}}

\linespread{1.3}

\def\labelitemi{---}
\setlist[itemize]{leftmargin=1.25cm, itemindent=0.55cm, itemsep=0.001bp}
\setlist{nolistsep} % Отсутствие отступов между элементами \enumerate и \itemize
%\setlist[enumerate]{leftmargin=1.25cm, itemindent=0.55cm}

\newcommand{\specialcell}[2][c]{%
	\begin{tabular}[#1]{@{}c@{}}#2\end{tabular}}

\frenchspacing



\newcommand{\aasection}[2] {
	\vspace{20mm}
	{\let\clearpage\relax \chapter{#1}\label{#2}}
}

\newcommand{\aaunnumberedsection}[2] {
	\vspace{20mm}
	\addcontentsline{toc}{chapter}{#1}
	{\let\clearpage\relax \chapter*{#1}\label{#2}}
}

\usepackage{pgfplots}
\pgfplotsset{compat=1.18}
\usepackage{pgfplots}
\pgfplotsset{compat=1.18}
%\renewcommand{\arraystretch}{1.5}
\usepackage{makecell}